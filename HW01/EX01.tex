\documentclass[12pt]{article}

\usepackage[utf8]{inputenc}
\usepackage{latexsym,amsfonts,amssymb,amsthm,amsmath}

\setlength{\parindent}{0in}
\setlength{\oddsidemargin}{0in}
\setlength{\textwidth}{6.5in}
\setlength{\textheight}{8.8in}
\setlength{\topmargin}{0in}
\setlength{\headheight}{18pt}



\title{ Sheet 1 }
\author{Our Names}

\begin{document}

\maketitle

\vspace{0.5in}



\subsection*{Exercise 1}
\begin{itemize}

\item[1.] Equation of the decision boundary $$0 = x_1 -2x_2 + 3x_3 - 1$$
The equation is a hyperplane since the result of this 3-dimensional equation is a 1 dimensionally lower, 2 dimensional surface which hence a hyperplane.
\item[2.] Normal vector n to the decision boundary 
\[n = w = \begin{bmatrix} 1 \\ -2 \\ 3 \end{bmatrix}\] 
Its Direction shows to the points of the class with y = 1.
\vspace{1cm} %Leave space for comments!
\item[3.]
\begin{itemize}
\item[(a)] \[
d(\mathbf{x}) = \frac{\mathbf{w}^\top \mathbf{x} + b}{\|\mathbf{w}\|} = \frac{1-2+3-1}{\sqrt{1^2+(-2)^2+3^2}} = \frac{1}{\sqrt{14}} \approx 0,267 
\]
\vspace{1cm} %Leave space for comments!
\item[(b)] If the sign of d is positive the output of y is going to be 1, if it's negative the output will be 0.
\vspace{1cm} %Leave space for comments!
\item[(c)] The higher the distance, the more confident the model is in its prediction.
\vspace{1cm} %Leave space for comments!
\item[(d)] \[
x - d(x) \ ||w|| \ w = \begin{bmatrix} 1 \\ 1 \\ 1 \end{bmatrix} - \frac{1}{14}  \begin{bmatrix} 1 \\ -2 \\ 3 \end{bmatrix} = \begin{bmatrix} \frac{13}{14} \\ \frac{16}{14} \\ \frac{11}{14} \end{bmatrix}
\]
\vspace{1cm} %Leave space for comments!
\item[(e)] 
\vspace{1cm} %Leave space for comments!
\end{itemize}
\end{itemize}

\vspace{2in} %Leave more space for comments!







\end{document}
